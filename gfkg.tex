\documentclass[11pt, a4paper]{article}

\usepackage{amsmath}
\usepackage{amssymb}
\usepackage{amsthm}
\usepackage[tracking=true, kerning=true]{microtype}
\usepackage{hyperref}
\usepackage[T1]{fontenc}
\usepackage{lmodern}
\usepackage{inconsolata}
\usepackage{parskip}

\DeclareMicrotypeAlias{lmss}{cmr}

\frenchspacing

\renewcommand*{\d}{\ensuremath{\mathrm{d}}}
\renewcommand*{\thesection}{\Roman{section}}
\newcommand*{\email}[1]{\normalsize\texttt{\href{mailto:#1}{#1}}}
\newcommand*{\norm}[1]{\ensuremath{\left\Vert#1\right\Vert}}

\theoremstyle{definition}
\newtheorem{ex}{Exercise}[section]
\newtheorem{sol}{Solution}[section]

\title{Gauge Fields, Knots and Gravity Solutions}
\author{Fionn Fitzmaurice}
\date{}

\makeatletter
\hypersetup{pdftitle = \@title,
            pdfauthor = \@author,
            pdfcreator = TeX,
            hidelinks,
            pdfpagemode = UseNone
}
\makeatother

\author{Fionn Fitzmaurice \hspace{20pt} \email{fionn@maths.tcd.ie}}

\begin{document}

\maketitle
\thispagestyle{empty}

\section{Electromagnetism}

\subsection{Maxwell's Equations}

\begin{ex}

Let $\vec{k}$ be a vector in $\mathbb{R}^3$ and let $\omega = |\vec{k}|$. Fix $\vec{E} \in \mathbb{C}^3$ with $\vec{k} \cdot \vec{E} = 0$ and $i \vec{k} \times \vec{E} = \omega \vec{E}$. Show that
\[
    \vec{\mathcal{E}}(t, \vec{x}) = \vec{E} e^{-i(\omega t - \vec{k} \cdot \vec{x})}
\]
satisfies the vacuum Maxwell equations.

\end{ex}

\begin{sol}

Recall that Maxwell's equations (where $c = 1$) are
\begin{align*}
    \nabla \cdot \vec{B} &= 0, \tag{M1}\\
    \nabla \times \vec{E} + \frac{\partial \vec{B}}{\partial t} &= 0, \tag{M2}\\
    \nabla \cdot \vec{E} &= \rho, \tag{M3} \\
    \nabla \times \vec{B} - \frac{\partial \vec{E}}{\partial t} &= \vec{\jmath}. \tag{M4}
\end{align*}
The vacuum equations are invariant under
\[
    \vec{B} \mapsto \vec{E}, \qquad \vec{E} \mapsto - \vec{B}
\]
(electromagnetic duality) or, equivalently, for a complex-valued vector field $\vec{\mathcal{E}} = \vec{E} + i \vec{B}$,
\[
    \vec{\mathcal{E}} \mapsto i \vec{\mathcal{E}}.
\]
This lets us express the vacuum equations in terms of $\vec{\mathcal{E}}$ as
\[
    \nabla \cdot \vec{\mathcal{E}} = 0, \qquad \nabla \times \vec{\mathcal{E}} = i \frac{\partial \vec{\mathcal{E}}}{\partial t}.
\]

We'll rely on
\[
    \partial_j e^{-i(\omega t - \vec{k} \cdot \vec{x})} = i k_j e^{-i(\omega t - \vec{k} \cdot \vec{x})}
\]
to show that our $\vec{\mathcal{E}}$ satisfies the vacuum equations.

For the divergence,
\begin{align*}
    \nabla \cdot \vec{\mathcal{E}} &= \nabla \cdot \left(\vec{E} e^{-i(\omega t - \vec{k} \cdot \vec{x})}\right) \\
        &= \sum_{j = 1}^3 \partial_j \left( E_j  e^{-i(\omega t - \vec{k} \cdot \vec{x})} \right) \\
        &= \sum_{j = 1}^3 E_j \partial_j e^{-i(\omega t - \vec{k} \cdot \vec{x})} \\
        &= \sum_{j = 1}^3 E_j i k_j e^{-i(\omega t - \vec{k} \cdot \vec{x})} \\
        &= i \vec{k} \cdot \vec{E} e^{-i(\omega t - \vec{k} \cdot \vec{x})} \\
        &= 0.
\end{align*}

For the curl (dropping the summation and using Einstein notation),
\begin{align*}
    {\left( \nabla \times \vec{\mathcal{E}} \right)}_i &= \epsilon_{ijk} \partial_j \mathcal{E}_k \\
        &= \epsilon_{ijk} \partial_j \left(E_k e^{-i(\omega t - \vec{k} \cdot \vec{x})} \right) \\
        &= \epsilon_{ijk} E_k \partial_j e^{-i(\omega t - \vec{k} \cdot \vec{x})} \\
        &= \epsilon_{ijk}i k_j E_k e^{-i(\omega t - \vec{k} \cdot \vec{x})} \\
        &= {\left(i \vec{k} \times \vec{E} e^{-i(\omega t - \vec{k} \cdot \vec{x})} \right)}_i \\
        &= \omega \vec{E}_i e^{-i(\omega t - \vec{k} \cdot \vec{x})} \\
        &= \omega \vec{\mathcal{E}}_i,
\end{align*}
so $\nabla \times \vec{\mathcal{E}} = \omega \vec{\mathcal{E}}$.
But
\begin{align*}
    \frac{\partial \vec{\mathcal{E}}}{\partial t} &= \frac{\partial}{\partial t} \left(\vec{E} e^{-i(\omega t - \vec{k} \cdot \vec{x})} \right) \\
        &= -i \omega \vec{E} e^{-i(\omega t - \vec{k} \cdot \vec{x})} \\
        &= -i \omega \vec{\mathcal{E}},
\end{align*}
giving
\[
    \nabla \times \vec{\mathcal{E}} = \omega \vec{\mathcal{E}} = i \frac{\partial \vec{\mathcal{E}}}{\partial t}
\]
and satisfying the second vacuum equation.

\end{sol}

\subsection{Manifolds}

A function $f: X \to Y$ from one topological space to another is defined to be continuous if, given any open set $U \subseteq Y$, the inverse image $f^{-1}(U) \subseteq X$ is open.

\begin{ex}

Show that a function $f: \mathbb{R}^n \to \mathbb{R}^m$ is continuous according to the above definition if and only if it is continuous according to the epsilon--delta definition: for all $x \in \mathbb{R}^n$ and all $\epsilon > 0$, there exists $\delta > 0$ such that $\norm{y - x} < \delta$ implies $\norm{f(y) - f(x)} < \epsilon$.

\end{ex}

\begin{sol}

Suppose $f$ is continuous according to the epsilon--delta definition of continuity.
Let $V \subseteq \mathbb{R}^m$ be an open set.
For any $x \in f^{-1}(V)$, since $f(x) \in V$ there exists a ball of radius $\epsilon$, $B(f(x), \epsilon) \subseteq V$, centered at $f(x)$.
Then by the epsilon--delta condition there exists a ball of radius $\delta$, $B(x, \delta) \subseteq \mathbb{R}^n$ such that
\[
    f(B(x, \delta)) \subset B(f(x), \epsilon).
\]
Since $x$ was arbitrary, $f^{-1}(V)$ is open as all points sufficiently close to $x$ are also in $f^{-1}(V)$.

Suppose $f$ is continuous according to the topological definition of continuity.
Let $x \in \mathbb{R}^n$ and $\epsilon > 0$.
Consider the open set $f^{-1}(B(f(x), \epsilon)) \subseteq \mathbb{R}^n$.
There exists a $\delta > 0$ such that
\[
    B(x, \delta) \subset f^{-1}(B(f(x), \epsilon)).
\]
Therefore for any point $y \in B(x,\delta)$, $f(y) \in B(f(x), \epsilon)$ or, equivalently, $\norm{y - x} < \delta$ implies $\norm{f(y) - f(x)} < \epsilon$.

\end{sol}

\begin{ex}

Given a topological space $X$ and a subset $S \subseteq X$, define the \emph{induced topology} on $S$ to be the topology in which the open sets are of the form $U \cap S$, where $U$ is open in $X$.

Let $S^n$, the $n$-sphere, be the unit sphere in $\mathbb{R}^{n + 1}$:
\[
    S^n = \biggl\{\vec{x} \in \mathbb{R}^{n + 1} \Bigm| \sum_{i = 1}^{n + 1} {(x^i)}^2 = 1 \biggr\}.
\]
Show that $S^n \subset \mathbb{R}^{n + 1}$ with its induced topology is a manifold.

\end{ex}

\begin{sol}

We need to show that:
\begin{itemize}
    \item the open sets of the induced topology $\{U_\alpha\}$ cover $S^n$,
    \item there exists an atlas of charts $\varphi_\alpha: U_\alpha \to \mathbb{R}^n$ for all $\alpha$,
    \item the transition functions $\varphi_\alpha \circ \varphi_\beta^{-1}: \mathbb{R}^n \to \mathbb{R}^n$ are smooth where defined (since we include ``smooth'' in our definition of a manifold).
\end{itemize}

Consider the sets
\[
    U_1 = S^n \setminus \left\{(0, \ldots, 0, 1)\right\}, \qquad
    U_{-1} = S^n \setminus \left\{(0, \ldots, 0, -1)\right\}
\]
which each exclude a single pole. Each $U_\alpha$ is of the form $U \cap S^n$ where $U$ is open in $\mathbb{R}^{n + 1}$.
The induced topology $\left\{U_1, U_{-1}\right\}$ is a cover of $S^n$.

Let $\varphi_\alpha: U_\alpha \to \mathbb{R}^n$ be the stereographic projection (for $\alpha \in \{-1, 1\})$.
For some $\vec{p} \in S^n$, $\varphi_\alpha(\vec{p}) \in \mathbb{R}^n$ should be a point on the line that intersects $S^n$ at $\vec{s}_\alpha = (0, \ldots, 0, \alpha)$.
Take a segment of this line parameterised by $t \in [0, 1]$ as
\begin{align*}
    (1 - t) \vec{s}_\alpha + t\vec{p} &= (t p_1, \ldots t p_n, \alpha(1 - t) + tp_{n + 1}) \\
        &= (t p_1, \ldots t p_n, \alpha + t(p_{n + 1} - \alpha)).
\end{align*}
This intersects $\mathbb{R}^n$ when the last coordinate $\alpha + t(p_{n + 1} - \alpha) = 0$, so $t = \frac{1}{1 - \alpha p_{n + 1}}$ and the projection is therefore given by
\[
    \varphi_\alpha: \vec{p} \mapsto \left(\frac{p_1}{1 - \alpha p_{n + 1}}, \ldots, \frac{p_n}{1 - \alpha p_{n + 1}}\right).
\]
Each projection is a chart and the collection of these charts is an atlas, since the union of their domains covers $S^n$.

Denoting $\varphi_\alpha(\vec{p}) = \vec{x}_\alpha = \left(x_\alpha^1, \ldots, x_\alpha^n\right)$, the $L\!^2$-norm
\begin{align*}
    r_\alpha^2 &= \sum_{i = 1}^n {(x_\alpha^i)}^2 \\
               &= \frac{p_1^2 + \cdots + p_n^2}{{(1 - \alpha p_{n + 1})}^2} \\
               &= \frac{1 - p_{n + 1}^2}{{(1 - \alpha p_{n + 1})}^2} \\
               &= \frac{(1 + p_{n + 1})(1 - p_{n + 1})}{{(1 - \alpha p_{n + 1})}^2} \\
               &= {\left(\frac{1 + p_{n + 1}}{1 - p_{n + 1}}\right)}^\alpha,
\end{align*}
so
\[
    p_{n + 1} = \alpha\frac{r_\alpha^2 - 1}{r_\alpha^2 + 1}.
\]
This gives us a general expression for the points $p_1, \ldots, p_n$ on the manifold in terms of our chart's coordinate system as
\begin{align*}
    p_i &= x_\alpha^i (1 - \alpha p_{n + 1}) \\
        &= \frac{2 x_\alpha^i}{r_\alpha^2 + 1},
\end{align*}
so the inverse projections $\varphi_\alpha^{-1}: \mathbb{R}^n \to S^n$ are given by
\[
    \varphi_\alpha^{-1}: \vec{x} \mapsto \left(\frac{2 x^1}{r^2 + 1}, \ldots, \frac{2 x^n}{r^2 + 1}, \alpha \frac{r^2 - 1}{r^2 + 1} \right).
\]
For inverse map $\varphi_\beta^{-1}$, note that the point $p_{n + 1}$ is given by
\[
    p_{n + 1} = \beta \frac{r^2 - 1}{r^2 + 1}.
\]
From this, and assuming $\alpha$, $\beta$ are distinct so $\alpha \beta = -1$, we get that
\[
    \frac{1}{1 - \alpha p_{n + 1}} = \frac{r^2 + 1}{2 r^2}.
\]
The transition functions $\varphi_\alpha \circ \varphi_\beta^{-1}: \mathbb{R}^n \to \mathbb{R}^n$ (with distinct $\alpha$, $\beta$) are then given by
\begin{align*}
    \varphi_\alpha \circ \varphi_\beta^{-1} (\vec{x}) &= \varphi_\alpha \left(\left(\frac{2 x^1}{r^2 + 1}, \ldots, \frac{2 x^n}{r^2 + 1}, \beta \frac{r^2 - 1}{r^2 + 1} \right)\right) \\
        &= \left(\frac{2 x^1}{r^2 + 1} \cdot \frac{r^2 + 1}{2 r^2},
                 \ldots,
                 \frac{2 x^n}{r^2 + 1} \cdot \frac{r^2 + 1}{2 r^2}
            \right) \\
        &= \frac{\vec{x}}{\norm{x}^2}.
\end{align*}
These transition functions are inversions on the $n$-sphere and are smooth where they are defined.

\end{sol}

\begin{ex}

Show that if $M$ is a manifold and $U$ is an open subset of $M$, then $U$ with its induced topology is a manifold.

\end{ex}

\begin{sol}

All subsets $U_\alpha \subset U$ are of the form $V \cap U$ where $V$ is open in $M$, so the open sets of the induced topology cover $U$.

We can construct an atlas by taking the charts on $M$, $\varphi_\alpha: V_\alpha \to \mathbb{R}^n$, and defining
\begin{align*}
    \varphi^U_\alpha&: U_\alpha \to \mathbb{R}^n, \\
                    &: u \mapsto \varphi_\alpha(u),
\end{align*}
i.e. $\varphi^U_\alpha = \varphi_\alpha$ for all $U_\alpha$.
Since $U_\alpha$ is open, we have well defined transition functions so $U$ with the induced topology is a manifold.

\end{sol}

\begin{ex}

Given topological spaces $X$ and $Y$, we give $X \times Y$ the \emph{product topology} in which a set is open if and only if it is a union of sets of the form $U \times V$, where $U$ is open in $X$ and $V$ is open in $Y$. Show that if $M$ is an $m$-dimensional manifold and $N$ is an $n$-dimensional manifold, $M \times N$ is an $(m + n)$-dimensional manifold.

\end{ex}

\begin{sol}

For every point $(u, v) \in M \times N$, there exists a set $U \times V$ where $U$ is open in $M$ and $V$ is open in $N$ such that $u \in U$, $v \in V$.
Therefore $U \times V$ is an open set under the product topology and $M \times N$ is a topological space.

Given $M$, $N$ are manifolds, they have atlases
\[
    \left\{\varphi^M_\alpha : U_\alpha \to \mathbb{R}^m \right\}, \qquad
    \left\{\varphi^N_\beta : V_\beta \to \mathbb{R}^n \right\}
\]
for all $U_\alpha$ open in $M$, $V_\beta$ open in $N$.

For some $u \in U_\alpha$, $v \in V_\beta$, denote
\[
    \varphi^M_\alpha: u \mapsto \vec{x} = (x_1, \ldots, x_m), \quad
    \varphi^N_\beta: v \mapsto \vec{y} = (y_1, \ldots, y_n).
\]

We can construct maps $\tilde{\varphi}_{\alpha\beta}:\ U_\alpha \times V_\beta \to \mathbb{R}^m \times \mathbb{R}^n$ as
\begin{align*}
    \tilde{\varphi}_{\alpha\beta} (u, v) &= \left(\varphi^M_\alpha(u), \varphi^N_\beta(v)\right) \\
        &= (\vec{x}, \vec{y}).
\end{align*}
This is obviously invertible via
\[
    \tilde{\varphi}_{\alpha\beta}^{-1}(\vec{x}, \vec{y}) = \left({(\varphi^M_\alpha)}^{-1}(\vec{x}), {(\varphi^N_\beta)}^{-1}(\vec{y})\right) = (u, v).
\]
because the inverse charts are guaranteed to exist.

The product space $\mathbb{R}^m \times \mathbb{R}^n$ is homeomorphic to $\mathbb{R}^{m + n}$ under
\[
    h(\vec{x}, \vec{y}) = (x_1, \ldots, x_m, y_1, \ldots, y_n),
\]
so we can construct new smooth maps $\varphi_{\alpha\beta} = h \circ \tilde{\varphi}_{\alpha\beta}$ that target $\mathbb{R}^{m + n}$.
The transition functions
\[
    \varphi_{\alpha\beta} \circ \varphi_{\alpha\beta}^{-1}: \mathbb{R}^{m + n} \to \mathbb{R}^{m + n}
\]
are similarly obviously smooth where defined, so $\varphi_{\alpha\beta}$ is a chart and the collection of these charts for all $U_\alpha$, $V_\beta$ is an atlas, therefore $M \times N$ is a manifold.

\end{sol}

\end{document}
