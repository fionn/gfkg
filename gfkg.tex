\documentclass[11pt, a4paper]{article}

\usepackage{amsmath}
\usepackage{amssymb}
\usepackage{amsthm}
\usepackage[tracking=true, kerning=true]{microtype}
\usepackage{hyperref}
\usepackage[T1]{fontenc}
\usepackage{lmodern}
\usepackage{inconsolata}
\usepackage{parskip}

\DeclareMicrotypeAlias{lmss}{cmr}

\frenchspacing

\renewcommand*{\d}{\ensuremath{\mathrm{d}}}
\renewcommand*{\thesection}{\Roman{section}}
\newcommand*{\email}[1]{\normalsize\texttt{\href{mailto:#1}{#1}}}

\theoremstyle{definition}
\newtheorem{ex}{Exercise}[section]
\newtheorem{sol}{Solution}[section]

\title{Gauge Fields, Knots and Gravity Solutions}
\author{Fionn Fitzmaurice}
\date{}

\makeatletter
\hypersetup{pdftitle = \@title,
            pdfauthor = \@author,
            pdfcreator = TeX,
            hidelinks,
            pdfpagemode = UseNone
}
\makeatother

\author{Fionn Fitzmaurice \hspace{20pt} \email{fionn@maths.tcd.ie}}

\begin{document}

\maketitle
\thispagestyle{empty}

\section{Electromagnetism}

\subsection{Maxwell's Equations}

\begin{ex}

Let $\vec{k}$ be a vector in $\mathbb{R}^3$ and let $\omega = |\vec{k}|$. Fix $\vec{E} \in \mathbb{C}^3$ with $\vec{k} \cdot \vec{E} = 0$ and $i \vec{k} \times \vec{E} = \omega \vec{E}$. Show that
\[
    \vec{\mathcal{E}}(t, \vec{x}) = \vec{E} e^{-i(\omega t - \vec{k} \cdot \vec{x})}
\]
satisfies the vacuum Maxwell equations.

\end{ex}

\begin{sol}

Recall that Maxwell's equations (where $c = 1$) are
\begin{align*}
    \nabla \cdot \vec{B} &= 0, \tag{M1}\\
    \nabla \times \vec{E} + \frac{\partial \vec{B}}{\partial t} &= 0, \tag{M2}\\
    \nabla \cdot \vec{E} &= \rho, \tag{M3} \\
    \nabla \times \vec{B} - \frac{\partial \vec{E}}{\partial t} &= \vec{\jmath}. \tag{M4}
\end{align*}
The vacuum equations are invariant under
\[
    \vec{B} \mapsto \vec{E}, \qquad \vec{E} \mapsto - \vec{B}
\]
(electromagnetic duality) or, equivalently, for a complex-valued vector field $\vec{\mathcal{E}} = \vec{E} + i \vec{B}$,
\[
    \vec{\mathcal{E}} \mapsto i \vec{\mathcal{E}}.
\]
This lets us express the vacuum equations in terms of $\vec{\mathcal{E}}$ as
\[
    \nabla \cdot \vec{\mathcal{E}} = 0, \qquad \nabla \times \vec{\mathcal{E}} = i \frac{\partial \vec{\mathcal{E}}}{\partial t}.
\]

We'll rely on
\[
    \partial_j e^{-i(\omega t - \vec{k} \cdot \vec{x})} = i k_j e^{-i(\omega t - \vec{k} \cdot \vec{x})}
\]
to show that our $\vec{\mathcal{E}}$ satisfies the vacuum equations.

For the divergence,
\begin{align*}
    \nabla \cdot \vec{\mathcal{E}} &= \nabla \cdot \left(\vec{E} e^{-i(\omega t - \vec{k} \cdot \vec{x})}\right) \\
        &= \sum_{j = 1}^3 \partial_j \left( E_j  e^{-i(\omega t - \vec{k} \cdot \vec{x})} \right) \\
        &= \sum_{j = 1}^3 E_j \partial_j e^{-i(\omega t - \vec{k} \cdot \vec{x})} \\
        &= \sum_{j = 1}^3 E_j i k_j e^{-i(\omega t - \vec{k} \cdot \vec{x})} \\
        &= i \vec{k} \cdot \vec{E} e^{-i(\omega t - \vec{k} \cdot \vec{x})} \\
        &= 0.
\end{align*}

For the curl (dropping the summation and using Einstein notation),
\begin{align*}
    {\left( \nabla \times \vec{\mathcal{E}} \right)}_i &= \epsilon_{ijk} \partial_j \mathcal{E}_k \\
        &= \epsilon_{ijk} \partial_j \left(E_k e^{-i(\omega t - \vec{k} \cdot \vec{x})} \right) \\
        &= \epsilon_{ijk} E_k \partial_j e^{-i(\omega t - \vec{k} \cdot \vec{x})} \\
        &= \epsilon_{ijk}i k_j E_k e^{-i(\omega t - \vec{k} \cdot \vec{x})} \\
        &= {\left(i \vec{k} \times \vec{E} e^{-i(\omega t - \vec{k} \cdot \vec{x})} \right)}_i \\
        &= \omega \vec{E}_i e^{-i(\omega t - \vec{k} \cdot \vec{x})} \\
        &= \omega \vec{\mathcal{E}}_i,
\end{align*}
so $\nabla \times \vec{\mathcal{E}} = \omega \vec{\mathcal{E}}$.
But
\begin{align*}
    \frac{\partial \vec{\mathcal{E}}}{\partial t} &= \frac{\partial}{\partial t} \left(\vec{E} e^{-i(\omega t - \vec{k} \cdot \vec{x})} \right) \\
        &= -i \omega \vec{E} e^{-i(\omega t - \vec{k} \cdot \vec{x})} \\
        &= -i \omega \vec{\mathcal{E}},
\end{align*}
giving
\[
    \nabla \times \vec{\mathcal{E}} = \omega \vec{\mathcal{E}} = i \frac{\partial \vec{\mathcal{E}}}{\partial t}
\]
and satisfying the second vacuum equation.

\end{sol}

\end{document}
